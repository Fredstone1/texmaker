%%%%%%%%%%%%%%%%%%%%%%%%%%%%%%%%%%%%%%%%%
% Programming/Coding Assignment
% LaTeX Template
%
% This template has been downloaded from:
% http://www.latextemplates.com
%
% Original author:
% Ted Pavlic (http://www.tedpavlic.com)
%
% Note:
% The \lipsum[#] commands throughout this template generate dummy text
% to fill the template out. These commands should all be removed when 
% writing assignment content.
%
% This template uses a Perl script as an example snippet of code, most other
% languages are also usable. Configure them in the "CODE INCLUSION 
% CONFIGURATION" section.
%
%%%%%%%%%%%%%%%%%%%%%%%%%%%%%%%%%%%%%%%%%

%----------------------------------------------------------------------------------------
%	PACKAGES AND OTHER DOCUMENT CONFIGURATIONS
%----------------------------------------------------------------------------------------
\documentclass{report}
\usepackage[utf8]{inputenc}
\usepackage{titlesec}
\usepackage{lmodern}
\usepackage[T1]{fontenc}
\usepackage{fancyhdr} % Required for custom headers
\usepackage{lastpage} % Required to determine the last page for the footer
\usepackage{extramarks} % Required for headers and footers
\usepackage[usenames,dvipsnames]{color} % Required for custom colors
\usepackage{graphicx} % Required to insert images
\usepackage{listings} % Required for insertion of code
\usepackage{courier} % Required for the courier font
\usepackage{lipsum} % Used for inserting dummy 'Lorem ipsum' text into the template

% Margins
\topmargin=-0.45in
\evensidemargin=0in
\oddsidemargin=0in
\textwidth=6.5in
\textheight=9.0in
\headsep=0.25in

\linespread{1.1} % Line spacing

% Set up the header and footer
\pagestyle{fancy}
\lhead{\hmwkAuthorName} % Top left header
\chead{} % Top center head
\rhead{\hmwkTitle} % Top right header
\lfoot{\hmwkClass} % Bottom left footer
\cfoot{} % Bottom center footer
\rfoot{Page\ \thepage\ of\ \protect\pageref{LastPage}} % Bottom right footer
\renewcommand\headrulewidth{0.4pt} % Size of the header rule
\renewcommand\footrulewidth{0.4pt} % Size of the footer rule
\setlength\parindent{0pt} % Removes all indentation from paragraphs

%----------------------------------------------------------------------------------------
%	CODE INCLUSION CONFIGURATION
%----------------------------------------------------------------------------------------



%----------------------------------------------------------------------------------------
%	DOCUMENT STRUCTURE COMMANDS
%	Skip this unless you know what you're doing
%----------------------------------------------------------------------------------------
\definecolor{RoyalRed}{RGB}{157,16, 45}
\titleformat{\chapter}[display]
  { \normalsize \huge  \color{black}}
  {\flushright \normalsize \color{RoyalRed} \MakeUppercase { \chaptertitlename \hspace{1 ex} }  { \fontsize{60}{60}\selectfont \color{RoyalRed} \thechapter }} {10 pt}{\bfseries\huge} 


%----------------------------------------------------------------------------------------
%	NAME AND CLASS SECTION
%----------------------------------------------------------------------------------------

\newcommand{\hmwkTitle}{SON - Self-Organizing Networks } % Assignment title
\newcommand{\hmwkDueDate}{Friday,\ December\ 8,\ 2017} % Due date
\newcommand{\hmwkClass}{Introduction to mobile communication -\ 34330 \\} % Course/class
\newcommand{\hmwkClassTime}{Department of Photonics Engineering} % Kursusansvarlig profession
\newcommand{\hmwkClassInstructor}{Henrik Lehrmann Christiansen,} % Teacher/lecturer
\newcommand{\hmwkAuthorName}{Frederik Rander Andersen, s164146} % Your name

%----------------------------------------------------------------------------------------
%	TITLE PAGE
%----------------------------------------------------------------------------------------

\title{
\vspace{-0.5in}
\noindent\makebox[\linewidth]{\rule{\textwidth}{1pt}} 
\textmd{\textbf{\hmwkClass \hmwkTitle}}\\
\normalsize\vspace{0.1in}\small{\hmwkDueDate}\\
\vspace{0.1in}\large{\textit{\hmwkClassInstructor\ \hmwkClassTime}}
\noindent\makebox[\linewidth]{\rule{\textwidth}{1pt}} 
\vspace{1in}
}

\author{\textbf{\hmwkAuthorName}}
\date{} % Insert date here if you want it to appear below your name

%----------------------------------------------------------------------------------------

\begin{document}
\begin{figure}
	\centering
	\includegraphics[scale=0.15]{dtulogo.png}
\end{figure}

\begin{figure}[!b]
	\includegraphics[scale=0.25]{dtufooter.png}
\end{figure}
\maketitle

%----------------------------------------------------------------------------------------
%	TABLE OF CONTENTS
%----------------------------------------------------------------------------------------

\newpage
\tableofcontents
\newpage

%----------------------------------------------------------------------------------------
%	INTRODUCTION & OVERVIEW
%----------------------------------------------------------------------------------------

\chapter{SON - Introduction and overview}
\section{Introduction}
All mobile networks need to be managed and as systems become more and more complex, the need for better and easier ways to manage them are important as ever. LTE (Long Term Evolution) is the newest technology and also the most complex. Therefore, in LTE, management needs to be as good as possible. SON (Self-Organizing Networks) is a very promising area for providers, as it makes managing networks cheaper, more efficient and easier. 
The goal of SON is basically to reduce the need for technicians and increase the network capabilities, such that the network will be as good as possible in regards to coverage, capacity and user experience. 


\section{Overview}

\section{Why SON?}
The reasons for using SON are very obvious from a provider standpoint. First of all, the cost of a Self-Organizing Network should be much lower 

	
	%----------------------------------------------------------------------------------------
%	SELF-CONFIGURATION
%----------------------------------------------------------------------------------------

\chapter{Self-configuration}


%----------------------------------------------------------------------------------------
%	SELF-OPTIMIZATION\cite{•}
%----------------------------------------------------------------------------------------

\chapter{Self-optimization}


%----------------------------------------------------------------------------------------
%	SELF-HEALING
%----------------------------------------------------------------------------------------

\chapter{Self-healing}


%----------------------------------------------------------------------------------------
%	BIBLIOGRAPHY
%----------------------------------------------------------------------------------------
\clearpage
\addcontentsline{toc}{chapter}{References}
\bibliographystyle{unsrt}
\bibliography{/home/frederik/Documents/bibtex/SON.bib}


\end{document}