%----------------------------------------------------------------------------------------
%	PACKAGES AND OTHER DOCUMENT CONFIGURATIONS
%----------------------------------------------------------------------------------------
\documentclass{report}
\usepackage[utf8]{inputenc}
\usepackage{titlesec}
\usepackage{lmodern}
\usepackage[T1]{fontenc}
\usepackage{fancyhdr} % Required for custom headers
\usepackage{lastpage} % Required to determine the last page for the footer
\usepackage{extramarks} % Required for headers and footers
\usepackage[usenames,dvipsnames]{color} % Required for custom colors
\usepackage{graphicx} % Required to insert images
\usepackage{listings} % Required for insertion of code
\usepackage{courier} % Required for the courier font
\usepackage{lipsum} % Used for inserting dummy 'Lorem ipsum' text into the template

% Margins
\topmargin=-0.45in
\evensidemargin=0in
\oddsidemargin=0in
\textwidth=6.5in
\textheight=9.0in
\headsep=0.25in

\linespread{1.1} % Line spacing

% Set up the header and footer
\pagestyle{fancy}
\lhead{\hmwkAuthorName} % Top left header
\chead{} % Top center head
\rhead{\hmwkTitle} % Top right header
\lfoot{\hmwkClass} % Bottom left footer
\cfoot{} % Bottom center footer
\rfoot{Page\ \thepage\ of\ \protect\pageref{LastPage}} % Bottom right footer
\renewcommand\headrulewidth{0.4pt} % Size of the header rule
\renewcommand\footrulewidth{0.4pt} % Size of the footer rule
\setlength\parindent{0pt} % Removes all indentation from paragraphs

%----------------------------------------------------------------------------------------
%	CODE INCLUSION CONFIGURATION
%----------------------------------------------------------------------------------------



%----------------------------------------------------------------------------------------
%	DOCUMENT STRUCTURE COMMANDS
%	Skip this unless you know what you're doing
%----------------------------------------------------------------------------------------
\definecolor{RoyalRed}{RGB}{157,16, 45}
\titleformat{\chapter}[display]
  { \normalsize \huge  \color{black}}
  {\flushright \normalsize \color{RoyalRed} \MakeUppercase { \chaptertitlename \hspace{1 ex} }  { \fontsize{60}{60}\selectfont \color{RoyalRed} \thechapter }} {10 pt}{\bfseries\huge} 


%----------------------------------------------------------------------------------------
%	NAME AND CLASS SECTION
%----------------------------------------------------------------------------------------

\newcommand{\hmwkTitle}{SON - Self-Organizing Networks } % Assignment title
\newcommand{\hmwkDueDate}{Friday,\ December\ 8,\ 2017} % Due date
\newcommand{\hmwkClass}{Introduction to mobile communication -\ 34330 \\} % Course/class
\newcommand{\hmwkClassTime}{Department of Photonics Engineering} % Kursusansvarlig profession
\newcommand{\hmwkClassInstructor}{Henrik Lehrmann Christiansen,} % Teacher/lecturer
\newcommand{\hmwkAuthorName}{Frederik Rander Andersen, s164146} % Your name

%----------------------------------------------------------------------------------------
%	TITLE PAGE
%----------------------------------------------------------------------------------------

\title{
\vspace{-0.5in}
\noindent\makebox[\linewidth]{\rule{\textwidth}{1pt}} 
\textmd{\textbf{\hmwkClass \hmwkTitle}}\\
\normalsize\vspace{0.1in}\small{\hmwkDueDate}\\
\vspace{0.1in}\large{\textit{\hmwkClassInstructor\ \hmwkClassTime}}
\noindent\makebox[\linewidth]{\rule{\textwidth}{1pt}} 
\vspace{1in}
}

\author{\textbf{\hmwkAuthorName}}
\date{} % Insert date here if you want it to appear below your name

%----------------------------------------------------------------------------------------

\begin{document}
\begin{figure}
	\centering
	\includegraphics[scale=0.15]{dtulogo.png}
\end{figure}

\begin{figure}[!b]
	\includegraphics[scale=0.25]{dtufooter.png}
\end{figure}
\maketitle

%----------------------------------------------------------------------------------------
%	TABLE OF CONTENTS
%----------------------------------------------------------------------------------------

\newpage
\tableofcontents
\newpage

%----------------------------------------------------------------------------------------
%	INTRODUCTION & OVERVIEW
%----------------------------------------------------------------------------------------

\chapter{SON - Introduction and overview}
\section{Introduction}
This report will be taking a closer look at SON (Self-Organizing Networks) and what it means for the industry of telecommunications. 
The features of SON aims to improve end user experience and reduce the costs entailed with providing a network, while still increasing the quality and effiency of the network. 
These features, along with their impact will be explored later in the report 

\section{Overview}
All mobile networks need to be managed and as systems become more and more complex, the need for better and easier ways to manage them are important as ever. LTE (Long Term Evolution) is the newest technology and also the most complex. Therefore, in LTE, management needs to be as good as possible. SON (Self-Organizing Networks) is a very promising area for providers, as it makes network-management cheaper, more efficient and easier. This is also why SON is most prevalent in LTE networks, simply because the demands of LTE are much higher and therefore LTE networks are quite complex.\\

The goal of SON is basically to reduce the need for technicians and increase the network capabilities, such that the network will be as good as possible in regards to coverage, capacity and user experience. Generally, SON has three main areas; self-configuration, self-optimization and self-healing. These will discussed in depth later. 

\section{Why SON?}
The reasons for using SON are very obvious from a provider standpoint. First of all, the cost of a Self-Organizing Network should be much lower 
	%----------------------------------------------------------------------------------------
%	SELF-CONFIGURATION
%----------------------------------------------------------------------------------------
\chapter{Self-configuration}
\section{Main idea and overview}
The first area of SON is self-configuration. The main idea behind the self-configuration part of SON is to automate the setup of eNBs (eNodeB). This allows a plug and play type of setup, which saves the network owner a lot of time and money, since you would usually need a technician to setup new eNBs, which could take a lot of time.  
The self-configuration also reduces the risk of incorrect installation and integration of eNBs into the existing network. The amount of needed cells is also rising with the increase in network usage.

\subsection{Process of eNB self-configuration}
The self-configuration process of a eNB starts with the new eNB receiving an IP address. It can now obtain the information of the self-configuration subsystem of operation and management. Next, the eNB will have a gateway configured, such that it will be able to communicate with other internet devices through the exchange of IP packets. \\
Now, the eNB provides all of its details e.g. hardware, type etc., to the self-configuration subsystem to be authenticated. The self-configuration subsystem will then provide the necessary software and configuration data to the eNB and the eNB will configure itself accordingly. \\
The eNB is now ready to connect to the operation and management system for management functions. Now S1 interface is established, meaning that the eNB is connected to the Evolved Packet Core Network. The X2 interface is also established by this point and the eNB is now connected to other eNBs in the network.

%----------------------------------------------------------------------------------------
%	SELF-OPTIMIZATION\cite{•}
%----------------------------------------------------------------------------------------
\chapter{Self-optimization}


%----------------------------------------------------------------------------------------
%	SELF-HEALING
%----------------------------------------------------------------------------------------
\chapter{Self-healing}


%----------------------------------------------------------------------------------------
%	BIBLIOGRAPHY
%----------------------------------------------------------------------------------------
\clearpage
\addcontentsline{toc}{chapter}{References}
\bibliographystyle{ieeetr}
\bibliography{/home/frederik/Documents/bibtex/SON.bib}
\nocite{Feng2008} % remove these if you cite
\nocite{Hamalainen2009} % remove these if you cite
\nocite{Jamalipour2008} % remove these if you cite
\nocite{Kakadia2017} % remove these if you cite
\nocite{Sartori2012} % remove these if you cite

\end{document}