%----------------------------------------------------------------------------------------
%	PACKAGES AND OTHER DOCUMENT CONFIGURATIONS
%----------------------------------------------------------------------------------------
\documentclass{report}
\usepackage[utf8]{inputenc}
\usepackage{titlesec}
\usepackage{lmodern}
\usepackage[T1]{fontenc}
\usepackage{fancyhdr} % Required for custom headers
\usepackage{lastpage} % Required to determine the last page for the footer
\usepackage{extramarks} % Required for headers and footers
\usepackage[usenames,dvipsnames]{color} % Required for custom colors
\usepackage{graphicx} % Required to insert images
\usepackage{listings} % Required for insertion of code
\usepackage{courier} % Required for the courier font
\usepackage{lipsum} % Used for inserting dummy 'Lorem ipsum' text into the template

% Margins
\topmargin=-0.45in
\evensidemargin=0in
\oddsidemargin=0in
\textwidth=6.5in
\textheight=9.0in
\headsep=0.25in

\linespread{1.1} % Line spacing

% Set up the header and footer
\pagestyle{fancy}
\lhead{\hmwkAuthorName} % Top left header
\chead{} % Top center head
\rhead{\hmwkTitle} % Top right header
\lfoot{\hmwkClass} % Bottom left footer
\cfoot{} % Bottom center footer
\rfoot{Page\ \thepage\ of\ \protect\pageref{LastPage}} % Bottom right footer
\renewcommand\headrulewidth{0.4pt} % Size of the header rule
\renewcommand\footrulewidth{0.4pt} % Size of the footer rule
\setlength\parindent{0pt} % Removes all indentation from paragraphs

%----------------------------------------------------------------------------------------
%	CODE INCLUSION CONFIGURATION
%----------------------------------------------------------------------------------------



%----------------------------------------------------------------------------------------
%	DOCUMENT STRUCTURE COMMANDS
%	Skip this unless you know what you're doing
%----------------------------------------------------------------------------------------
\definecolor{RoyalRed}{RGB}{157,16, 45}
\titleformat{\chapter}[display]
  { \normalsize \huge  \color{black}}
  {\flushright \normalsize \color{RoyalRed} \MakeUppercase { \chaptertitlename \hspace{1 ex} }  { \fontsize{60}{60}\selectfont \color{RoyalRed} \thechapter }} {10 pt}{\bfseries\huge} 


%----------------------------------------------------------------------------------------
%	NAME AND CLASS SECTION
%----------------------------------------------------------------------------------------

\newcommand{\hmwkTitle}{SON - Self-Organizing Networks } % Assignment title
\newcommand{\hmwkDueDate}{Friday,\ December\ 8,\ 2017} % Due date
\newcommand{\hmwkClass}{Introduction to mobile communication -\ 34330 \\} % Course/class
\newcommand{\hmwkClassTime}{Department of Photonics Engineering} % Kursusansvarlig profession
\newcommand{\hmwkClassInstructor}{Henrik Lehrmann Christiansen,} % Teacher/lecturer
\newcommand{\hmwkAuthorName}{Frederik Rander Andersen, s164146} % Your name

%----------------------------------------------------------------------------------------
%	TITLE PAGE
%----------------------------------------------------------------------------------------

\title{
\vspace{-0.5in}
\noindent\makebox[\linewidth]{\rule{\textwidth}{1pt}} 
\textmd{\textbf{\hmwkClass \hmwkTitle}}\\
\normalsize\vspace{0.1in}\small{\hmwkDueDate}\\
\vspace{0.1in}\large{\textit{\hmwkClassInstructor\ \hmwkClassTime}}
\noindent\makebox[\linewidth]{\rule{\textwidth}{1pt}} 
\vspace{1in}
}

\author{\textbf{\hmwkAuthorName}}
\date{} % Insert date here if you want it to appear below your name

%----------------------------------------------------------------------------------------

\begin{document}
\begin{figure}
	\centering
	\includegraphics[scale=0.15]{dtulogo.png}
\end{figure}

\begin{figure}[!b]
	\includegraphics[scale=0.25]{dtufooter.png}
\end{figure}
\maketitle

%----------------------------------------------------------------------------------------
%	TABLE OF CONTENTS
%----------------------------------------------------------------------------------------

\newpage
\tableofcontents
\newpage

%----------------------------------------------------------------------------------------
%	INTRODUCTION & OVERVIEW
%----------------------------------------------------------------------------------------

\chapter{SON - Introduction and overview}
\section{Introduction}
This report will be taking a closer look at SON (Self-Organizing Networks) and what it means for the industry of telecommunications. 
The features of SON aims to improve end user experience and reduce the costs entailed with providing a network, while still increasing the quality and efficiency of the network. 
These features, along with their impact will be explored later in the report 

\section{Overview}
All mobile networks need to be managed and as systems become more and more complex, the need for better and easier ways to manage them are important as ever. LTE (Long Term Evolution) is the newest technology and also the most complex. Therefore, in LTE, management needs to be as good as possible. SON (Self-Organizing Networks) is a very promising area for providers, as it makes network-management cheaper, more efficient and easier. This is also why SON is most prevalent in LTE networks, simply because the demands of LTE are much higher and therefore LTE networks are quite complex.\\ The goal of SON is basically to reduce the need for technicians and increase the network capabilities, such that the network will be as good as possible in regards to coverage, capacity and user experience. Generally, SON has three main areas; self-configuration, self-optimization and self-healing. These will discussed in depth later. 

\section{Why SON?}
The reasons for using SON are very obvious from a provider standpoint. First of all, the cost of a Self-Organizing Network should be much lower 
	%----------------------------------------------------------------------------------------
%	SELF-CONFIGURATION
%----------------------------------------------------------------------------------------
\chapter{Self-configuration}
\section{Main idea and overview}
The first area of SON is self-configuration. The main idea behind the self-configuration part of SON is to automate the setup of eNBs (eNodeB). This allows a plug and play type of setup, which saves the network owner a lot of time and money, since you would usually need a technician to setup new eNBs, which could take a lot of time.  
The self-configuration also reduces the risk of incorrect installation and integration of eNBs into the existing network. The amount of needed cells is also rising with the increase in network usage.

\section{Features of self-configuration}
There are three main features of self-configuration in LTE; self-configuration of eNB, Automatic Neighbour Relations and automatic configuration of Physical Cell ID (PCI). 

Another small but important feature of self-configuration is Dynamic Radio Configuration (DNC). 
In order for the eNB to configure itself correctly, it will need to alter the planned data a little. This is done automatically by the eNB itself by performing measurements and thus adjusting initial power, antenna tilt etc. %citer http://www.radio-electronics.com/info/cellulartelecomms/self-organising-networks-son/self-configuration.php


\subsection{Process of eNB self-configuration}

%following section might be deleted
%As described above, an eNB will configure itself, and it will generally perform the following steps in order to be able to communicate:\\ The self-configuration process of a eNB starts with the new eNB receiving an IP address. It can now obtain the information of the self-configuration subsystem of operation and management. Next, the eNB will have a gateway configured, such that it will be able to communicate with other internet devices through the exchange of IP packets. \\
%Now, the eNB provides all of its details e.g. hardware, type etc., to the self-configuration subsystem to be authenticated. The self-configuration subsystem will then provide the necessary software and configuration data to the eNB and the eNB will configure itself accordingly. \\ The eNB is now ready to connect to the operation and management system for management functions. Now S1 interface is established, meaning that the eNB is connected to the Evolved Packet Core Network. The X2 interface is also established by this point and the eNB is now connected to other eNBs in the network.\cite{Feng2008} %EVT INKLUDER FIGUR 1 fra den citerede tekst
%end of section
One of the key ideas of self-configuration is self-configuration of a new eNB trying to connect to the network. The eNB is in this case not connected to anything but the network management subsystem and Serving-Gateway (SGW). The following steps are performed in order to connect the new eNB to the network. 

\begin{itemize}
	\item First, the eNB is powered on and plugged in where necessary, then it will have an already 	established connection, which it will use until the radio frequency transmission is turned on.
	\item The DNS/DHCP server will now provide an IP address to the eNB.
	\item Now, the self-configuration subsystem of the of operation and management information is 		sent to the eNB. 
	\item A gateway is now configured to the eNB such that it can now connect and communicate with other internet nodes through IP packets.
	\item The eNB will provide its own information e.g. hardware, ID, supported technologies etc. to the self-configuration subsystem in order to get identified and authenticated. 
	\item The eNB will now be able to download the correct software and radio-configuration information. 
	\item After this download, the eNB configures itself according to the downloaded transport and radio configuration information. 
	\item Now, the eNB can connect to the Operation Administration Management (OAM) for other management functions.
	\item The S1 interface is then established, giving the eNB connection to the Evolved Packet Core Network (EPCN). The X2 interface is also established and the eNB is now connected to other eNBs in the network. \cite{Atayero2014}
\end{itemize}

\subsection{Automatic Neighbour Relations (ANR)}

Automatic Neighbour Relation (ANR), is the process of managing the neighbour relations, such that the cells know who their neighbouring cells are and what technologies they support. It is crucial that this information is updated and is correct, otherwise handovers might fail and thereby result in dropped calls. If the neighbouring cells support different technologies e.g. one might support HSPA and its neighbour LTE, the user needs to know this so it know what frequencies to listen on. \\ Generally the ANR process looks a bit like this:

\begin{itemize}
	\item UE detects unknown PCI and reports it to the serving eNB by sending a Radio Resource Controller (RRC) reconfiguration message.
	\item The serving eNB will now request the UE to send E-UTRAN Cell Global ID (ECGI) of the unknown eNB. The UE then does this by reading the BCCH. 
	\item The serving eNB will now retrieve the IP address from the MME based on the ECGI of the unknown eNB. This allows the serving eNB to correctly setup the X2 interface, since that hasn't been done, because the unknown eNB is unknown. 
	\item The functions will now be extended to cases of inter-Radio Access Technology (inter-RAT) and inter-frequency. \cite{Atayero2014}
\end{itemize}

\subsection{Automatic configuration of Physical Cell ID (PCI)}
Every cell in the network will have its own PCI, which is in the SCH, to be used when synchronizing with UE on downlink. In the E-UTRAN there are only 504 PCIs, therefore they have to be reused, however they need to be unique within specific regions. There are two rules which has to be obeyed when configuring PCIs: 
Neighbouring cells must not share PCI. 
All neighbours of one cell must have different PCIs. \cite{Feng2008}


%----------------------------------------------------------------------------------------
%	SELF-OPTIMIZATION
%----------------------------------------------------------------------------------------
\chapter{Self-optimization}
\section{Main idea and overview}
After the network has self-configured it should also be able to adapt according to load and other factors. This is where self-optimization comes in. To ensure effective management and operation of the network, the network should, after configuring itself, have algorithms and processes to make decisions in regards to efficiency and performance. If implemented correctly this will increase performance and stability while keeping cost and energy at a minimum. This is done by taking measurements from both UEs and eNBs and using these to regulate the network. 

\section{Features of self-optimization}
There are several important features of self-optimization, which all contribute to making SON much better. 

\subsection{Mobility Load Balancing (MLB)}
Mobile Load Balancing (MLB) is especially relevant in cases where a cell with high load and a cell with low load are neighbours. Here MLB uses automated algorithms and functions to avoid cell overloading and consequent degradation of performance. This is achieved by having the algorithms adjust parameters to balance the load between cells, these algorithms also prevent other issues that could arise in these situations e.g. ping pong handover. 
Simply put, MLB in eNBs tries to balance load relatively evenly between neighbouring cells, improve network capacity by regulating cell congestions and manage the network efficiently such that performance is as high as possible.
MLB consists of three main parts when used between cells; load reporting, load balancing action and amending mobility configuration.
Here is an example of how MLB could work:
\begin{itemize}
	\item eNB A detects that it is being overloaded
	\item eNB A sends a Resource Status Request to its neighbour eNB B
	\item eNB B answers with a Resource Status response
	\item eNB B sends a Resource Status Update
	\item eNB A will now find suitable UEs to handover to eNB B
	\item Now the handover procedure begins
	\item eNB A sends a Mobility Change Request to eNB B (to change eNB As range)
	\item eNB B answers with a Mobility Change ACK
	\item both eNBs now updates their handover settings to ensure that the UEs won't go back to the previously overloaded eNB A
\end{itemize}

\subsection{Mobility Robustness Optimization (MRO)}
Mobility Robustness Optimization (MRO) is mainly used to guarantee good mobility for UE and thereby giving the end-users a good experience. Furthermore, it is used to ensure that handovers are as seamless as possible. Previously, a network operator would have to manually adjust handover parameters in cells, which is very costly, MRO automates this process, meaning lower costs for the network operator and better experience for the end-user. \\ It is important to mention that call re-establishment is only possible in LTE, so if the target cell only supports different Radio Access Technologies (RATs), it will not be possible to restore the connection. 
Another thing that MRO tries to do is minimizing handover ping pong. While ping pong does not lead to RLF, it is however, an inefficient use of resources and could reduce the users throughput. 

\subsubsection{Backward Handover}
Backward handover occurs when handover information is exchanged between the eNB and the UE on the old radio path, instead of the new radio path between the target eNB and the UE. Then the old eNB, also called source eNB, prepares the target eNB for handover. So, in backward handover, it is the network that perform cell switching and notifies the UE of the target eNB. 
Backward handover is used when the Radio Frequency conditions are declining.

\subsubsection{Radio Link Failure (RLF) handover}
The way MRO achieves all this is by minimizing call drops and RLF. Sometimes a connection can be restored after a RLF but before the call is dropped. This is known as RLF handover, and occurs when a backward handover partially fails. The process of RLF handover looks a bit like this: 
\begin{itemize}
	\item UE detects radio link problems and starts its RLF timer.
	\item RLF timer expires and the UE now searches for a new target eNB and it now attempts to re-establish the connection with the new eNB. This is all done while the UE is in a connected state. 
	\item The connection is restored if the UE managed to send a measurement report to the old eNB, such that the target eNB could be prepared.  %citer qualcomm whitepaper
\end{itemize}
It should be noted that no data is lost in this procedure due to the eNB forwarding all data contained in its buffer. 
Sometimes, the mobility problems are so severe that the call is dropped. 

\subsection{Minimization of Drive Tests (MDT)}
Testing is very important for network operators, as it helps to collect statistics about the network. A lot of the testing done on the the network is drive tests, which consists of sending out engineers in a car, then they drive around and test the network in different areas. This requires a lot of special tools, it also takes a long time and is quite costly. Therefore, MDT is very important to network operators. The idea behind MDT is to use the UEs to measure and log information along with location, just like the drive tests. Because of the amount of users and their different behaviour this method of testing would be very thorough, and would give the network operators an exact view of what their customers are experiencing. The main use of MDT would be to optimize network coverage e.g. by detecting network "holes". 

\subsection{Energy Savings}

\subsection{Coverage and Capacity Optimization}

\subsection{RACH optimization}
%----------------------------------------------------------------------------------------
%	SELF-HEALING
%----------------------------------------------------------------------------------------
\chapter{Self-healing}


%----------------------------------------------------------------------------------------
%	Conclusion and summary
%----------------------------------------------------------------------------------------
\chapter{Conclusion}




%----------------------------------------------------------------------------------------
%	BIBLIOGRAPHY
%----------------------------------------------------------------------------------------
\clearpage
\addcontentsline{toc}{chapter}{References}
\bibliographystyle{ieeetr}
\bibliography{/home/frederik/Documents/bibtex/SON.bib}
\nocite{Hamalainen2009} % remove these if you cite
\nocite{Jamalipour2008} % remove these if you cite
\nocite{Kakadia2017} % remove these if you cite
\nocite{Sartori2012} % remove these if you cite

\end{document}